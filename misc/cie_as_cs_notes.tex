\documentclass[a4paper]{article}
\usepackage[margin=1in]{geometry}


\title{CIE AS Level CS'24 Notes | Chapters 1 \& 2}
\author{Information Representation and Communication}
\date{Raufoon}


\begin{document}
\maketitle
\section{Information Representation}
\subsection{Data Representation}
Candidates should be able to:
\begin{itemize}
  \item Show understanding of binary magnitudes and the difference between binary prefixes and decimal prefixes.
  \item Show understanding of different number systems.
  \item Perform binary addition and subtraction.
  \item Describe practical applications where Binary Coded Decimal (BCD) and Hexadecimal are used.
  \item Show understanding of and be able to represent character data in its internal binary form, depending on the character set used.
\end{itemize}

Notes and guidance:
\begin{itemize}
  \item Understand the difference between and use: kibi and kilo, mebi and mega, gibi and giga, tebi and tera.
  \item Use the binary, denary, hexadecimal number bases and Binary Coded Decimal (BCD) and one's and two's complement representation for binary numbers.
  \item Convert an integer value from one number base/representation to another.
  \item Using positive and negative binary integers.
  \item Show understanding of how overflow can occur.
\end{itemize}

Students are expected to be familiar with ASCII, extended ASCII, and Unicode.

\subsection{Multimedia Graphics}
Candidates should be able to:
\begin{itemize}
  \item Show understanding of how data for a bitmapped image are encoded.
  \item Show understanding of the effects of changing elements of a bitmap image on the image quality and file size.
  \item Show understanding of how data for a vector graphic are encoded.
  \item Justify the use of a bitmap image or a vector graphic for a given task.
\end{itemize}

\subsubsection{Sound}
Candidates should be able to:
\begin{itemize}
  \item Show understanding of how sound is represented and encoded.
  \item Show understanding of the impact of changing the sampling rate and resolution.
\end{itemize}

\subsection{Compression}
Candidates should be able to:
\begin{itemize}
  \item Show understanding of the need for and examples of the use of compression.
  \item Show understanding of lossy and lossless compression and justify the use of a method in a given situation.
  \item Show understanding of how a text file, bitmap image, vector graphic, and sound file can be compressed, including the use of run-length encoding (RLE).
\end{itemize}

\section{Communication}
\subsection{Networks including the internet}
Candidates should be able to:
\begin{itemize}
  \item Show understanding of the purpose and benefits of networking devices.
  \item Show understanding of the characteristics of a LAN (local area network) and a WAN (wide area network).
  \item Explain the client-server and peer-to-peer models of networked computers.
  \item Show understanding of thin-client and thick-client and the differences between them.
  \item Show understanding of the bus, star, mesh, and hybrid topologies.
  \item Show understanding of cloud computing.
  \item Show understanding of the differences between and implications of the use of wireless and wired networks.
  \item Describe the hardware that is used to support a LAN.
  \item Describe the role and function of a router in a network.
  \item Show understanding of Ethernet and how collisions are detected and avoided.
  \item Show understanding of bit streaming.
  \item Show understanding of the differences between the World Wide Web (WWW) and the internet.
  \item Describe the hardware that is used to support the internet.
\end{itemize}

\end{document}
