\documentclass[a4paper]{article}
\usepackage[margin=1in]{geometry}

\title{CIE AS Level Computer Science '24 Notes | Chapters 1 \& 2}
\author{Information Representation and Communication}
\date{Raufoon}

\begin{document}
\maketitle

\section{Information Representation}
\subsection{Data Representation}
Candidates should be able to:
\subsubsection{Show understanding of binary magnitudes and the difference between binary prefixes and decimal prefixes.}
\begin{enumerate}
  \item Explain the concept of binary magnitudes and provide an example.
  \item Differentiate between binary prefixes and decimal prefixes, providing specific examples for each.
  \item Calculate the value of 1 kibi in binary notation.
  \item How is the use of binary prefixes different from decimal prefixes in the context of data storage?
\end{enumerate}

\subsubsection{Show understanding of different number systems.}
\begin{enumerate}
  \item Describe the binary number system and its significance in computing.
  \item Explain the octal and hexadecimal number systems and their applications.
  \item How does the binary number system differ from the decimal number system?
  \item Convert the decimal number 47 to its binary equivalent.
\end{enumerate}

\subsubsection{Perform binary addition and subtraction.}
\begin{enumerate}
  \item Perform binary addition for the numbers 1101 and 1011.
  \item Perform binary subtraction for the numbers 11010 and 1011.
  \item Calculate the result of adding 1111 and 1010 in binary.
\end{enumerate}

\subsubsection{Describe practical applications where Binary Coded Decimal (BCD) and Hexadecimal are used.}
\begin{enumerate}
  \item Explain the application of Binary Coded Decimal (BCD) in computing or electronics.
  \item Describe scenarios where hexadecimal representation is more advantageous than decimal representation.
  \item How is BCD used in the context of real-time clocks or timers?
\end{enumerate}

\subsubsection{Show understanding of and be able to represent character data in its internal binary form, depending on the character set used.}
\begin{enumerate}
  \item Explain the concept of character encoding and its significance.
  \item Describe the ASCII character set and provide an example of its usage.
  \item Demonstrate how the character "A" is represented in binary according to the ASCII standard.
  \item Differentiate between ASCII, extended ASCII, and Unicode character encoding.
\end{enumerate}

\subsection{Multimedia Graphics}
Candidates should be able to:
\subsubsection{Show understanding of the effects of changing elements of a bitmap image on the image quality and file size.}
\begin{enumerate}
  \item Explain how changing the resolution of a bitmap image affects its quality and file size.
  \item Describe the impact of altering the color depth (bit depth) on the quality and file size of a bitmap image.
  \item Provide examples of scenarios where reducing image quality is acceptable due to file size constraints.
\end{enumerate}

\subsubsection{Show understanding of how data for a vector graphic are encoded.}
\begin{enumerate}
  \item Explain the encoding method used for vector graphics.
  \item Differentiate between raster (bitmap) and vector graphics, highlighting their encoding differences.
  \item Describe how scalable vector graphics (SVG) are encoded and their advantages.
\end{enumerate}

\subsubsection{Justify the use of a bitmap image or a vector graphic for a given task.}
\begin{enumerate}
  \item Explain when it is more appropriate to use a bitmap image rather than a vector graphic.
  \item Justify the choice of vector graphics for creating logos or illustrations.
  \item Describe a scenario where the use of a bitmap image is the best choice.
\end{enumerate}

\subsection{Sound}
Candidates should be able to:
\subsubsection{Show understanding of how sound is represented and encoded.}
\begin{enumerate}
  \item Explain the representation of sound using analog and digital data.
  \item Define sampling and sampling rate in the context of sound representation.
  \item Describe how analog sound is converted into digital format.
\end{enumerate}

\subsubsection{Show understanding of the impact of changing the sampling rate and resolution.}
\begin{enumerate}
  \item Explain how increasing the sampling rate affects the quality and file size of a digital audio recording.
  \item Describe the impact of changing the resolution (bit depth) on the accuracy and file size of a sound recording.
  \item Justify the choice of a specific sampling rate for recording music.
\end{enumerate}

\subsection{Compression}
Candidates should be able to:
\subsubsection{Show understanding of the need for and examples of the use of compression.}
\begin{enumerate}
  \item Explain the purpose and significance of data compression in computing.
  \item Provide real-world examples of situations where data compression is essential.
  \item Describe the benefits of data compression in terms of storage and transmission.
\end{enumerate}

\subsubsection{Show understanding of lossy and lossless compression and justify the use of a method in a given situation.}
\begin{enumerate}
  \item Define lossy compression and its application in scenarios where some data loss is acceptable.
  \item Explain lossless compression and situations where preserving data integrity is critical.
  \item Justify the choice between lossy and lossless compression based on specific use cases.
\end{enumerate}

\subsubsection{Show understanding of how a text file, bitmap image, vector graphic, and sound file can be compressed, including the use of run-length encoding (RLE).}
\begin{enumerate}
  \item Explain the principles of compressing text files and provide an example.
  \item Describe the techniques and methods used to compress bitmap images.
  \item Explain how vector graphics can be compressed to reduce file size.
  \item Describe the use of Run-Length Encoding (RLE) in compressing data and provide an RLE example.
\end{enumerate}

\section{Communication}
\subsection{Networks including the internet}
Candidates should be able to:
\subsubsection{Show understanding of the purpose and benefits of networking devices.}
\begin{enumerate}
  \item Explain the purpose and role of a router in a network.
  \item Describe the function and benefits of using network switches in a LAN.
  \item Differentiate between a modem and a router and explain their distinct purposes.
\end{enumerate}

\subsubsection{Show understanding of the characteristics of a LAN (local area network) and a WAN (wide area network).}
\begin{enumerate}
  \item Define a LAN and its typical size and scope.
  \item Explain the characteristics and geographical scope of a WAN.
  \item Describe situations where a LAN is more suitable than a WAN and vice versa.
\end{enumerate}

\subsubsection{Explain the client-server and peer-to-peer models of networked computers.}
\begin{enumerate}
  \item Define the client-server model and provide examples of its use.
  \item Describe the peer-to-peer model and its application in a network environment.
  \item Differentiate between client-server and peer-to-peer network architectures.
\end{enumerate}

\subsubsection{Show understanding of thin-client and thick-client and the differences between them.}
\begin{enumerate}
  \item Explain the concept of a thin client and its role in a networked environment.
  \item Describe what a thick client is and when it is more appropriate to use one.
  \item Compare and contrast thin-client and thick-client models in terms of advantages and disadvantages.
\end{enumerate}

\subsubsection{Show understanding of the bus, star, mesh, and hybrid topologies.}
\begin{enumerate}
  \item Define and describe the bus network topology.
  \item Explain the characteristics and benefits of a star network topology.
  \item Describe the mesh network topology and its application in specific scenarios.
  \item Explain the concept of a hybrid network topology and provide examples of its use.
\end{enumerate}

\subsubsection{Show understanding of cloud computing.}
\begin{enumerate}
  \item Explain the concept of cloud computing and its significance.
  \item Describe the different service models in cloud computing (IaaS, PaaS, SaaS).
  \item Explain the benefits and potential drawbacks of using cloud services.
\end{enumerate}

\subsubsection{Show understanding of the differences between and implications of the use of wireless and wired networks.}
\begin{enumerate}
  \item Differentiate between wired and wireless network connections.
  \item Describe the implications of using a wired network in terms of reliability and speed.
  \item Explain the advantages and challenges of wireless networks in terms of mobility and security.
\end{enumerate}

\subsubsection{Describe the hardware that is used to support a LAN.}
\begin{enumerate}
  \item List and describe the essential hardware components of a local area network (LAN).
  \item Explain the role of a network interface card (NIC) in a LAN.
  \item Describe the purpose and function of a wireless access point (WAP) in a LAN.
\end{enumerate}

\subsubsection{Describe the role and function of a router in a network.}
\begin{enumerate}
  \item Explain the role of a router in directing data packets in a network.
  \item Describe how a router connects multiple networks and forwards data between them.
  \item Explain the concept of network address translation (NAT) and its role in a router.
\end{enumerate}

\subsubsection{Show understanding of Ethernet and how collisions are detected and avoided.}
\begin{enumerate}
  \item Explain what Ethernet is and its importance in network communications.
  \item Describe the Carrier Sense Multiple Access/Collision Detection (CSMA/CD) method used to detect and manage collisions.
  \item Explain how CSMA/CD helps ensure efficient data transmission in Ethernet networks.
\end{enumerate}

\subsubsection{Show understanding of bit streaming.}
\begin{enumerate}
  \item Define bit streaming and its relevance in media streaming.
  \item Describe the methods of bit streaming, including real-time and on-demand streaming.
  \item Explain the importance of bit rates in the context of media streaming.
\end{enumerate}

\subsubsection{Show understanding of the differences between the World Wide Web (WWW) and the internet.}
\begin{enumerate}
  \item Differentiate between the World Wide Web (WWW) and the internet.
  \item Explain the structure and function of the World Wide Web as a subset of the internet.
  \item Describe how the WWW relies on the internet's infrastructure for connectivity.
\end{enumerate}

\subsubsection{Describe the hardware that is used to support the internet.}
\begin{enumerate}
  \item List and describe the key hardware components that support the functioning of the internet.
  \item Explain the role of data centers and servers in providing internet services.
  \item Describe the importance of internet exchange points (IXPs) in global internet connectivity.
\end{enumerate}

\end{document}
